%% LyX 2.3.1-1 created this file.  For more info, see http://www.lyx.org/.
%% Do not edit unless you really know what you are doing.
\documentclass[english,openany]{scrbook}
\usepackage{fontspec}
\usepackage[a4paper]{geometry}
\geometry{verbose,tmargin=2cm,bmargin=2cm,lmargin=2.5cm,rmargin=2cm,headheight=1cm,headsep=0.5cm,footskip=1cm}
\usepackage{fancyhdr}
\pagestyle{fancy}
\setcounter{secnumdepth}{3}
\setcounter{tocdepth}{3}
\usepackage[unicode=true,
 bookmarks=true,bookmarksnumbered=false,bookmarksopen=false,
 breaklinks=false,pdfborder={0 0 0},pdfborderstyle={},backref=false,colorlinks=false]
 {hyperref}
\hypersetup{pdftitle={ZAP: Z-language Assembly Program},
 pdfauthor={Joel M. Berez},
 pdfkeywords={Infocom ZAP Zork IF IntFiction Interactive Fiction}}

\makeatletter

%%%%%%%%%%%%%%%%%%%%%%%%%%%%%% LyX specific LaTeX commands.
\providecommand\textquotedblplain{%
  \bgroup\addfontfeatures{Mapping=}\char34\egroup}

\@ifundefined{date}{}{\date{}}
\makeatother

\usepackage{listings}
\lstset{xleftmargin=0pt}
\usepackage{polyglossia}
\setdefaultlanguage[variant=american]{english}
\renewcommand{\lstlistingname}{Listing}

\begin{document}
\title{ZAP}
\subtitle{ZAP: Z-language Assembly Program}
\author{Joel M. Berez}
\publishers{INFOCOM INTERNAL DOCUMENT - NOT FOR DISTRIBUTION}

\maketitle
\tableofcontents{}

\chapter{Introduction}

ZAP is a two (or three) pass absolute assembler for Z-code. The intention
is for ZAP to provide relatively low-level support for all features
available in ZIP (Z-language Interpreter Program).

A ZAP program consists of a single file containing line-oriented statements.
A statement may produce code and/or data or may simply direct the
assembly process.

During the first pass, the assembler checks statement syntax, calculates
code/data locations, and attempts to resolve symbolic references.
During the second pass, code is generated, any error messages are
displayed, and an optional listing is produced. Due to the nature
of this two-pass process, certain symbol definition restrictions exist
that will be explained in the appropriate sections.

An optional prepass may be invoked to build the frequently used word
table by omitting the table (\texttt{FWORDS}) from the input file.
This is a table of substrings that may be inserted into larger strings
in an abbreviated form. The assembler will identify any such substrings
(defined by \texttt{.FSTR}) in strings that it assembles and use the
proper format. This optional prepass will search all regular strings,
find 32 good choices for substrings, and define them with \texttt{.FSTR}s
in the \texttt{FWORDS} table. See the ZIP manual for further details
concerning the string format.

Code that is generated by the assembler is directly executable by
ZIP as long as the program follows all necessary conventions for the
interpretor. ZAP will to a large extent insure that required data
structures are defined, ZIP pointers are initialized, etc. However
it cannot be responsible for such implementation dependent restrictions
as maximum table sizes. 

\chapter{Language Syntax}

\section{Character Set}

Except within strings or comments, only the following characters are
allowed in a Z program:

\begin{lstlisting}[basicstyle={\ttfamily}]
    A-Z           Symbol constituents
    0-9,-         Symbol or number constituents
    ?,#,.         Symbol constituents

    <space>,<tab> Ignored except as initial operand prefix
    <cr>,<ff>     Ignored
    <lf>          End-of-line character

    ,             General operand prefix
    >             Return value operand prefix
    /             Branch (on success) operand prefix
    \             Branch (on failure) operand prefix
    =             Value operand prefix (for assignments)
    +             Addend operand prefix (addition of constants)

    "             String delimiter

    ;             Comment prefix
\end{lstlisting}


\section{Symbols}

Symbols are used to represent values of various types. Because the
assembler associates a type with each symbol along with its value
when it is defined, no special method is required to specify the type
when the symbol is used. However, certain naming conventions are suggested
for the convenience of the programmer.

A symbol contains any positive number of characters from the set \texttt{\{A-Z,0-9,?,\#,.,-\}}.
By convention pseudo-ops, and only pseudo-ops, begin with a period.
Other conventions are left to the discretion of the programmer.

Symbols may be global or local depending upon the type of value that
is assigned to them. The range of a global symbol is the entire program,
while the range of a local symbol is restricted to the function in
which it is defined. While local symbols may be reused from one function
to another, only constants (always global) may be truly redefined.

The global symbols include two predefined sets. Operators are the
\textquotedbl hardware\textquotedbl{} Z-machine instructions and
use the same mnemonics shown in the ZIP documentation. Pseudo-ops
are used somewhat like operators, but are simply assembler directives
that may or may not generate code/data. They are explained later in
this manual.

Global labels can refer to either global data or to functions. (In
fact data, strings and tables, must be defined globally.) These are
defined either through the colon-colon construct, by a pseudo-op,
or by assignment to another global label. A few special globals are
predefined by the assembler (e.g.~\texttt{VOCAB} refers to the vocabulary
table).

Constants refer to arbitrary user-specified values. These are defined
by a direct assignment or by a pseudo-op. Unlike other symbols, constants
may be redefined at any time.

Global variables are defined by the \texttt{.GVAR} pseudo-op within
the \texttt{GLOBAL} table. They may be used interchangeably with local
variables, including the special \texttt{STACK} variable, and represent
unique data locations within the program.

Local symbols come in two varieties. Local variables are defined by
the \texttt{.FUNCT} pseudo-op, which defines a function and allocates
storage for its arguments and other local variables. These variables
are allocated on the stack when the function is called and are for
each such call assigned the initial values specified.

Local labels are the targets of branching instructions and are defined
by the colon construct or by assignment to another local label. 

\section{Statement Syntax}

A statement consists of the following four fields, all of which are
optional:

\begin{lstlisting}[basicstyle={\ttfamily}]
    <label> <operator> <operands> <comment>
\end{lstlisting}

A \texttt{<label>} is a single symbol terminated by either one or
two colons, depending upon whether the symbol is defined to be local
or global.

An \texttt{<operator>} is a pre-defined symbol from either the set
of operators or the set of pseudo-ops. It is terminated by the beginning
of another field or by the end-of-line.

The \texttt{<operands>} field begins with either a space or a tab,
unless it is the first field on the line. It may contain one or more
operands. Each operand after the first begins with an operand prefix
character. The operands field may be continued to the next line by
putting the prefix character on one line and the corresponding operand
on the next line. Each line may contain a comment.

The \texttt{<comment>} begins with a semi-colon and ends at the end-of-line.
It may contain any ASCII characters with the obvious exception of
line-feed.

\chapter{Pseudo-ops}

\section{Meta-syntax}

Below each of the assembler pseudo-ops is described. The syntax to
be used is shown first followed by a description. A pseudo-op statement
consists of the pseudo-op name in the operator field followed by zero
or more operands. A label may or may not be appropriate and a comment
is always allowed.

The meta-syntax used for describing the statement format shows the
pseudo-op followed by operand types, enclosed in angle brackets. Square
brackets are used to enclose optional operands. Operand types that
may be repeated zero or more times are followed by an ellipsis and
enclosed in braces.

The following operand types are used: 
\begin{description}
\item [{<number>}] An integer between -32768 and 65536. Numbers larger
than 32767 or smaller than 0 may be interpreted as either positive
or negative, depending upon the machine instruction.
\item [{<constant>}] A fixed user-defined value. May be a \texttt{<number>}.
\item [{<short~constant>}] A \texttt{<constant>} with a non-negative value
less than 256.
\item [{<long~constant>}] A \texttt{<constant>} that is not a \texttt{<short
constant>}.
\item [{<symbol>}] A symbol as described in section 2.2.
\item [{<pointer>}] A symbol that refers to a disk location, probably defined
with one of the colon constructs.
\item [{<any>}] Any \texttt{<symbol>} or \texttt{<number>}.
\item [{<string>}] Any number of characters of any ASCII value enclosed
in double-quotes. A double-quote may be included in the string by
using two consecutive double-quotes.
\item [{<short~string>}] A \texttt{<string>} not requiring more than 255
words to represent.
\end{description}

\section{Simple Data-generation Pseudo-ops}

\begin{lstlisting}[basicstyle={\ttfamily}]
.WORD <any>{,<any>...}
\end{lstlisting}

Generates the two-byte value of each \texttt{<any>}. 

For convenience, if \texttt{<any>} is written by itself, it will be
interpreted as \texttt{.WORD <any>}.

\medskip{}

\begin{lstlisting}[basicstyle={\ttfamily}]
.BYTE <short constant>{,<short constant>...}
\end{lstlisting}

Generates the one-byte values.

\medskip{}

\begin{lstlisting}[basicstyle={\ttfamily}]
.TRUE
\end{lstlisting}

Equivalent to \texttt{.WORD 1}.

\medskip{}

\begin{lstlisting}[basicstyle={\ttfamily}]
.FALSE
\end{lstlisting}

Equivalent to \texttt{.WORD 0}.

\pagebreak{}

\section{String Handling Pseudo-ops}

\begin{lstlisting}[basicstyle={\ttfamily}]
.ZWORD <string>
\end{lstlisting}

Generates the four-byte value of \texttt{<string>}, left-justified
and padded with spaces if necessary. \texttt{<string>} may not require
more than two words.

\medskip{}

\begin{lstlisting}[basicstyle={\ttfamily}]
.STR <string>
\end{lstlisting}

Generates \texttt{<string>} in two-byte words. The last word has the
end-of-string bit set and, if necessary, is padded with shift5 characters.
For convenience, if \texttt{<string>} is written without an operator,
it will be interpreted as \texttt{.STR <string>}.

\medskip{}

\begin{lstlisting}[basicstyle={\ttfamily}]
.FSTR <string>
\end{lstlisting}

Generates a string for the frequently used word table (\texttt{FWORDS}).
First does a \texttt{.STR <string>}, except that \texttt{<string>}
is not searched for fword substrings. Then adds the string to the
table of fword substrings. All \texttt{.FSTR}s should be in the 32-word
table following \texttt{FWORDS::}.

\medskip{}

\begin{lstlisting}[basicstyle={\ttfamily}]
.LEN <short string>
\end{lstlisting}

Generates a byte containing the number of words required to represent
the \texttt{<short string>}.

\medskip{}

\begin{lstlisting}[basicstyle={\ttfamily}]
.STRL <short string>
\end{lstlisting}

Equivalent to:

\begin{lstlisting}[basicstyle={\ttfamily}]
.LEN <short string>
.STR <short string>
\end{lstlisting}


\section{Assignment Pseudo-ops}

\begin{lstlisting}[basicstyle={\ttfamily}]
.EQUAL <symbol>,<any>
\end{lstlisting}

Assigns to \texttt{<symbol>} the same value and type as \texttt{<any>}.
If any is a \texttt{<number>}, the type becomes \texttt{<constant>}.
For convenience, the short form of \texttt{<symbol>=<any>} will also
be accepted by the assembler.

\medskip{}

\begin{lstlisting}[basicstyle={\ttfamily}]
.SEQ <symbol>{,<symbol>...}
\end{lstlisting}

The symbols are assigned as constants with sequential values beginning
with zero.

\medskip{}


\section{Special Purpose Pseudo-ops}

\begin{lstlisting}[basicstyle={\ttfamily}]
.TABLE [<number>] 
\end{lstlisting}

Declares that a table is being generated. Optionally specifies the
maximum length in bytes.

\medskip{}

\begin{lstlisting}[basicstyle={\ttfamily}]
.PROP <length>,<property> 
\end{lstlisting}

Generates a one-byte property header. \texttt{<length>} is a \texttt{<constant>}
between 1 and 8. \texttt{<property>} is a \texttt{<constant>} between
1 and 31.

\medskip{}

\begin{lstlisting}[basicstyle={\ttfamily}]
.ENDT 
\end{lstlisting}

Ends generation of the current table. If \texttt{<number>} was specified
in the \texttt{.TABLE} statement, ensures that not more than \texttt{<number>}
bytes have been used.

\pagebreak{}

\begin{lstlisting}[basicstyle={\ttfamily}]
.OBJECT <symbol>,<number1>,<number2>,<object1>,
        <object2>,<object3>,<pointer>
\end{lstlisting}

Generates an object with the specified elements. \texttt{<symbol>}
will be the object name and is assigned to the next available object
number. \texttt{<number1>} and \texttt{<number2>} are the flag words.
\texttt{<object1>}, \texttt{<object2>}, and \texttt{<object3>} are
object symbols refering to the \texttt{<loc>}, \texttt{<first>}, and
\texttt{<next>} pointers, respectively. \texttt{<pointer>} points
to the property table.

Note: all objects must be defined together in the \texttt{OBJECT}
table.

\medskip{}

\begin{lstlisting}[basicstyle={\ttfamily}]
.GVAR <symbol>[=<any>] 
\end{lstlisting}

Defines a new global variable named \texttt{<symbol>} and assigns
\texttt{<any>} as the default value. \texttt{<any>} defaults to zero.

Note: all global variables must be defined together in the \texttt{GLOBAL}
table.

\medskip{}

\begin{lstlisting}[basicstyle={\ttfamily}]
.FUNCT <symbol>{,<symbol>[=<any>]...} 
\end{lstlisting}

Begins generation of a function and starts a new local symbol block.
The first symbol is the function name. Any other symbols specified
become local variables. Default values may be given or will default
to zero.

\medskip{}


\section{Flow Control Pseudo-ops}

\begin{lstlisting}[basicstyle={\ttfamily}]
.INSERT <string> 
\end{lstlisting}

Logically inserts the contents of file \texttt{<string>} into the
current program at this point.

\medskip{}

\begin{lstlisting}[basicstyle={\ttfamily}]
.ENDI 
\end{lstlisting}

Ends the current \texttt{.INSERT} file and returns control to the
previous input source. Everything after this in the file will be ignored.

\medskip{}

\begin{lstlisting}[basicstyle={\ttfamily}]
.END 
\end{lstlisting}

Signifies the end of the program. Everything after this in the input
source will be ignored.

\chapter{Program Structure}

\section{Special Labels}

To satisfy the requirements of ZIP, pointers to certain locations
will automatically be assembled into a table at the beginning of the
program. (See the ZIP manual for the exact format.) To this end, certain
symbols must be defined by the user as global labels at the appropriate
positions in the program. The following symbols are required:
\begin{description}
\item [{VOCAB::}] Vocabulary table.
\item [{OBJECT::}] Object table.
\item [{GLOBAL::}] Global symbol table. This table must be the proper length,
as specified in the ZIP manual.
\item [{FWORDS::}] Frequently used word table. If omitted, will be automatically
generated by the assembler.
\item [{PURBOT::}] Beginning of pure (read-only) data and code.
\item [{ENDLOD::}] End of the preloaded data and code.
\item [{START::}] First instruction to be executed when game is started.
Must actually point to an instruction, not a function. 
\end{description}

\section{Program Order}

The program should be arranged in the following order:
\begin{enumerate}
\item \texttt{GLOBAL}, modifiable tables, and other impure data.
\item \texttt{PURBOT::}
\item \texttt{VOCAB}, \texttt{OBJECT}, \texttt{FWORDS}, and other pure,
preloaded tables, strings, and functions.
\item \texttt{ENDLOD::}
\item Non-preloaded tables, strings, and functions.
\item \texttt{.END}
\end{enumerate}
\texttt{START::} must be specified in front of some instruction.
\end{document}
